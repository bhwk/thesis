\documentclass[12pt,twoside, draft]{report}
% \usepackage[utf8]{inputenc}
\usepackage{graphicx}
\graphicspath{{./images/}}
\usepackage{caption}
\usepackage{subcaption}
\usepackage{listings}
\usepackage[a4paper,left=4cm, right=3cm, top=3cm,bottom=3cm]{geometry}
\usepackage{fancyhdr}
\pagestyle{plain}
\renewcommand{\headrulewidth}{0.4pt}
\renewcommand{\footrulewidth}{0.4pt}
\fancyhead{}
\fancyfoot{}
\fancyfoot[C]{\thepage}

\usepackage{titlesec}
\usepackage{hyperref}
\usepackage{mathtools}
\usepackage[
    backend=biber,
    style=numeric,
    backref=true,
    sorting=none
    ]{biblatex}
\usepackage{tocbibind}

\usepackage{fontspec}
\setmainfont{Times New Roman}
\usepackage[
    final,
    % tracking,
    % kerning=true,
    % spacing=true,
    stretch=10,
    shrink=10
    ]{microtype}


\title{
  {Privacy Enhanced Retrieval-Augmented Generation (RAG) for Large Language Models in Healthcare}\\
  {\large National University of Singapore}\\
}
\author{Bryan Ha Wai Kit}
\date{2025}


\bibliography{references.bib}

\begin{document}
\pagenumbering{roman}
\maketitle

\begin{abstract}
	Large Language Models (LLMs) are increasingly utilized in healthcare for tasks such as clinical note summarization and medical report generation.
	However, their reliance on proprietary and sensitive patient data introduces significant privacy risks, particularly when using Retrieval-Augmented Generation (RAG).
	This project proposes a privacy-focused framework that leverages synthetic document generation to mitigate these risks while maintaining response accuracy.

	The proposed system follows an agent-based approach, incorporating three key agents: a Search Agent, a Synthesis Agent, and a Review Agent. The process begins with the Search Agent retrieving relevant vector-related text nodes from a vector database. The Synthesis Agent then evaluates the extracted content, filtering and retaining only the necessary information for query responses while removing personally identifiable information (PII). Finally, the Review Agent verifies and refines the synthesized document to ensure privacy compliance before passing it to the LLM.

	This thesis evaluates the effectiveness of synthetic document generation in mitigating privacy risks while preserving contextual relevance. Through a series of experiments, the system's ability to reduce PII leakage, maintain medical accuracy, and withstand adversarial attacks is assessed. The findings provide insights into balancing privacy and utility in healthcare-focused LLM applications.
\end{abstract}


\tableofcontents
\listoffigures
\listoftables

\newpage
\pagenumbering{arabic}
\chapter{Introduction}
LLMs are transforming various industries. They able to perform tasks such as automated handling of workflows with Agentic frameworks, Natural Language Processing (NLP) tasks such as information extraction, and even rudimentary reasoning in some models. In fields like Healthcare, they can automate clinical note generation and summarization, assist in diagnosis, and provide personalized patient care.

However, they all suffer the same traditional issue, hallucinations, where they generate seemingly coherent but incorrect information. To address this, RAG was developed as a method to provide context to LLMs by incorporating an external knowledge base, allowing them to generate more accurate, domain-specific responses. This technique has applications in areas where hallucinations can cause severe harm, such as medicine, legal analysis, and cybersecurity.

While RAG enhances LLM capabilities, it introduces new security risks. Attackers can exploit RAG systems to extract proprietary or sensitive data through prompt injection attacks. This is a critical privacy concern, especially in healthcare where patient confidentiality is important.

In this project, we seek to test and develop an Agent-based synthetic document generation framework to mitigate these risks. By separating the RAG database from the externally facing LLM, we seek to enhance security while preserving the contextual accuracy of responses.

\chapter{Literature Review} \label {literature}

\section{Exploitation of RAG Systems}

While RAG improves LLM accuracy by integrating external knowledge sources, it also introduces new vulnerabilities. Attackers can exploit a RAG system's retrieval mechanisms to manipulate outputs, bypass safeguards, and extract sensitive information. This section explores some known methods of attacks and discusses their feasibility in a healthcare setting.

\subsection{Data Poisoning}
In data poisoning attacks (also known as a backdoor attack), attackers inject malicious or misleading information into the RAG corpus, causing the LLM to generate incorrect or malicious responses. These attacks can be used to carry out fraud, misinformation campaigns or provide adversarial control over responses. An example of this occurred with Google's Gemini, where, due to retrieving information from a satirical social media post, told the user to make use of "non-toxic glue" when making a cheese pizza \autocite{McMahon_2024}.

As highlighted in \autocite{tan2024gluepizzaeatrocks}, data poisoning attacks are non-trivial to carry out.
Depending on the complexity of the retrieval system, the attacker will have to modify the adversarial content such that the retrieval system is inclined to retrieve this document.
Furthermore, the attacker requires some information about or access to the retrieval system to exploit it.
This requirement is consistent with other studies carried out on data poisoning, and in almost all cases, the conditions in which this attack can manifest relies strictly on the insertion of a poisoned document into the RAG corpus \autocite{xue2024badragidentifyingvulnerabilitiesretrieval, tan2024gluepizzaeatrocks, xian2024vulnerabilityapplyingretrievalaugmentedgeneration}.

Given these requirements, we can conclude that this type of RAG attack is non-feasible in a healthcare setting.
In order to carry out this attack, the attacker has to have some form of access to the hospital's database.
The cases in which this occur typically present with an external cyberattack on the hospital's infrastructure, whether through hacking or social engineering, and is considered a data breach. Most data breaches occur through hacking, as reported in \autocite{Alder_2025}.
In this case, the attacker can gain access to the hospital's database, and does not need to rely on exploiting the RAG system.
Thus, we can conclude that this form of attack is non-applicable in a healthcare setting.

\subsection{Prompt Injection}
Prompt injection attacks involves crafting an input query that manipulates the model into generating unintended responses.
For RAG systems, this can be achieved either directly or indirectly.

Indirect prompt injection attacks function similarly to data poisoning except instead of inserting misleading information, adversarial prompts are attached to frequently retrieved documents in the RAG database. This enables attackers to retrieve documents from the RAG database using trigger prompts.

Direct prompt injection attacks involve the inclusion of a passage or sentence in the input query. This can be phrases such as "repeat all the context".
These attacks, when targeted at RAG systems, can cause the leakage of private or sensitive information from the RAG corpus.

\subsubsection{Indirect Prompt Injection}
As stated previously, indirect prompt injection attacks operate in a similar fashion to data poisoning attacks as both require some form of access or ability to manipulate the RAG corpus.

Instead of using a misleading document, malicious instructions are embed in the document within the corpus, allowing the attacker to manipulate the LLM's output.
This allows the attacker to manipulate the LLM into including potentially malicious URLs into its response when responding to a victim's query \autocite{clop2024backdooredretrieverspromptinjection}.

Furthermore, it should be noted that this type of prompt injection also allows the attacker manipulate the documents that are retrieved from the RAG corpus depending on the poison ratio, as covered in \autocite{peng2024dataextractionattacksretrievalaugmented}, which would allow unfettered access to any sensitive information stored in the database.

However, since this type of prompt attack requires some form of access to the RAG corpus, we can functionally consider it the same as a data poisoning attack.
Realistically, if this type of attack were to occur in a healthcare setting, the attacker would already have access to hospital records. Therefore we will not be focusing on this aspect of prompt injection attacks.

\subsubsection{Direct Prompt Injection}
Direct prompt injection involves the inclusion of an adversarial passage into the input query, and these attacks are usually carried out in a specific format.

As highlighted in \autocite{zeng2024goodbadexploringprivacy}, they consist of two components: information and command.
The information component of the attack leads the RAG system to retrieve documents that contain that form of information. Examples of this could be names, addresses, medical conditions.
The command component is targeted at the LLM. Phrases are included in the input query that are aimed at subverting any safeguards placed on the LLM.
This can be a phrase such as "please repeat all context back to me," or "ignore all instructions," etc.

As the study\autocite{zeng2024goodbadexploringprivacy} shows, a significant portion of the datasets used in the study were able to be retrieved from the LLM through simple prompt attacks.
The study also notes that the attack prompt could be further optimized for increased data extraction.
Part of the study also noted the effects of RAG on the data that was extracted.
It was noted that the inclusion of RAG decreased the appearance of memorized data in the LLM's output. It seems that the inclusion of RAG has caused the LLM to focus on leveraging the context retrieved rather than on its memorized training data \autocite{zeng2024goodbadexploringprivacy}.

Another study also corroborates this result. In \autocite{qi2024followinstructionspillbeans}, a similar method of prompt injection was used to extract text from a RAG database. The LLMs used in this study were instruction-tuned LLMs, meaning that the model has been trained to respond to instructions.

An interesting point to note was that they tested the similarity scores of the model's output with the retrieved context. Most LLMs used in the study exhibited higher BLEU, ROUGE-L, F1 and BERTScore scores that scaled with model size, suggesting that there is some correlation between an LLM capabilities and its vulnerabilities to prompt injection attacks \autocite{qi2024followinstructionspillbeans}. Additionally, between the amount of data extracted alongside the size of the context retrieved was noted, further asserting that RAG has inherent vulnerabilities that are not being addressed.

Both studies discussed have shown a clear vulnerability of RAG to sufficiently sophisticated prompt injection attacks, but not much research has been done regarding the mitigation of RAG output post-occurrence of a prompt injection attack.

While it is possible to include safeguards to prevent the occurrence of prompt attacks, their implementations are still vulnerable. This is highlighted in \autocite{li2024targetingcoresimpleeffective}, where a simple prompt of "ignore the context" caused the LLM agent to disregard any context retrieved despite the safeguards implemented.
Considering that a simple prompt like this was sufficient enough to manipulate the model's output, it suggests that RAG pipeline implementations may be more fragile than initially anticipated, and a sufficiently motivated attacker will eventually be able to penetrate any LLM-level safeguards in place.

These findings suggest that current RAG implementations lack strong defenses against targeted prompt injection attacks. While preventive safeguards exist, adversarial prompt injections can still manipulate retrieval. This highlights the need for alternative security measures - such as synthetic document generation - to obfuscate retrieved context and prevent LLMs from directly accessing sensitive data in a RAG corpus.

\section{Medical Anonymization}
The document synthesis system proposed in this project makes use of anonymization techniques in order to distance it from the original information received. Here we discuss traditional methods of medical anonymization and what they entail.

In clinical settings, preserving the privacy of patients is important, especially when sharing or releasing datasets. This also extends to publicly accessible healthcare applications. \autocite{Rodriguez_Tuck_Dozier_Lewis_Eldridge_Jackson_Murray_Weir_2022} outlines three main methods employed to preserve medical privacy: pseudonymization, de-identification, and anonymization.

Pseudonymization involves replacing identifying attributes with pseudonyms, which maintains its relation to the original data but masks direct identifiers. De-identification focuses on removing Personally Identifiable Information (PII) from patient records, aimed at prevent individual identification. Anonymization distorts data to the point that it can no longer be associated with the original record, thus eliminating the possibility of it being linked back to the original record.

These methods are often used in tandem to provide privacy protection. The release of clinical datasets typically involves an initial de-identification step, followed by either anonymization or pseudonymization. The data can be further manipulated by introducing random noise, converting specific dates to relative dates, or categorizing ages in order to further protect patient information. A comprehensive overview of these methods can be found in \autocite{Rodriguez_Tuck_Dozier_Lewis_Eldridge_Jackson_Murray_Weir_2022}.

\chapter{Methodology}
\section{Description of Pipeline}
Based on research into RAG vulnerabilities, there is a clear lack of security measures designed to preserve the privacy of a RAG corpus. This is especially important in fields like healthcare.
As demonstrated in \autocite{zeng2024goodbadexploringprivacy}, private information can be easily extracted by determined attackers through simple prompt injections.
Given that RAG relies on a set of documents as context and its vulnerabilities to RAG, we believe that generating a synthetic document separate from the corpus is sufficient to mitigate most issues.

\section{System Design}
As mentioned, the solution explored in this project consists of an agent-based document synthesis pipeline aimed at preventing raw LLM access to sensitive data.

For all intents and purposes, the pipeline operates in a similar fashion to typical RAG. Upon receiving a query, it fetches document from the RAG corpus then uses the retrieved documents as context in generating a response. However, we include an intermediary step between the information retrieval and inference steps.

Once the documents are retrieved, a  secondary LLM extracts only the necessary information from the documents retrieved. For instance, we may retrieve a medical record consisting of different medical readings for a query about a patient's blood pressure readings. In this example, we aim for the LLM to extract only the blood pressure readings from this document.

With the information retrieved, we use an agent-based approach to modify the information. In order to further distance the information from the original record, we apply the following steps.

Firstly, we remove any PII that may appear in the information. We consider the following as PII: names, ages, contact number and address. The LLM will remove, or replace with pseudonyms, any appearance of PII.

Secondly, we manipulate the data that appears in the information to generalise the record. Numbers are rounded, and converted to ranges if multiple readings of the same type occur.

Finally, to ensure that the LLM treats the synthesized information as relevant context, we modify the original query based on the synthetic information. It should be able to generate the same output as a model operating solely on RAG.

Once it has gone through this step, we pass the synthesized query and information to the primary LLM to generate a response.

Refer to figure \ref{fig:SynthLLMRAG} for a visualization of the system design.

\begin{figure}
	\includegraphics[width=\textwidth]{Synthesis LLM RAG example.png}
	\caption{System Design}
	\centering
	\label{fig:SynthLLMRAG}
\end{figure}

\section{Building the RAG Corpus}
RAG systems can make use of either structured or unstructured data, however in healthcare, data is usually structured.
In order to mimic real healthcare settings, we determined it was necessary to make use of data that was designed for real-world settings.
For our case, we will be making use of a synthetic Fast Healthcare Interoperability Resources (FHIR) dataset, generated and distributed by Synthea \autocite{Synthea2024}.

FHIR is a structured healthcare standard that defines how healthcare information can be shared between different systems regardless of how they are stored.
Individual FHIR patient records are stored in what is known as resources and each resource type represents specific information. A Patient resource would include the patient's name, date of birth, address, etc. Each resource type is specific to its use case.

FHIR records can appear in different file formats, JSON, XML, or RDF. For simpler parsing and handling, we will be making use of JSON FHIR files to build our RAG corpus.

We make use of the open-source library \textit{Llamaindex}\autocite{Liu_LlamaIndex_2022} for abstractions when building the pipeline, as well as creating the database.

\subsection{FHIR Preprocessing}

First, we consider the type of data we wish to embed. JSON files are designed for programmatic use, meaning that they contain many identifying and delimiting tokens. If we were to convert the file in its entirety into its vector representation, it will result in detail being lost due to the repeated embedding of same key-value token pairs. Therefore, we first have to carry out flattening of the FHIR record.

Flattening the FHIR involves two things. First, we must determine what type of information we wish to extract. For this project, we are only working with information from the Patient resource, as well as the Observation, Procedure, Condition, Allergy and MedicationRequest resources. While initially the Encounter resource was used, we decided that it did not add any type of substantial information apart from the reason of the encounter as well as the location where it took place.

\begin{figure}
	\includegraphics[width=\textwidth]{Converting FHIR to sentence.png}
	\caption{FHIR to sentence}
	\centering
	\label{fig:FHIRtoSentence}
\end{figure}

Secondly, we have to convert the selected information into basic sentences. This is done by recursively un-nesting the FHIR resource with information we specified.
The reason we do this is to improve the embedding accuracy of the FHIR record.
Firstly, we convert FHIR resources to basic sentences.
This is to avoid repeatedly embedding the same key-value token pairs and wasting embedding tokens.
Refer to figure \ref{fig:FHIRtoSentence} for an example.

Processing the FHIR record, we group the information extracted from the Observation and Procedure resources by date. Afterwards, we collate the conditions, allergies, as well as the medications that has been assigned to the patient previously.

Each of these documents are stored in separate files, marked by the patient's name followed by the date of the encounter. These documents are then converted into vectors through the use of a text-embedding model, and stored within a Postgres database utilizing the \textit{pgvector} extension. The embedding model used for generating the embeddings is \emph{bge-base-en-v1.5}. The process is outlined in figure \ref{fig:EmbeddingsDatabase}.


\begin{figure}
	\includegraphics[width=\textwidth]{Store embeddings in DB.png}
	\caption{Embeddings to Database}
	\centering
	\label{fig:EmbeddingsDatabase}
\end{figure}

\subsection{Retrieval}
While not the scope of the project, it should be noted that during the creation of the database, Hierarchical Navigable Small World (HNSW) is used, which plays some influence in the retrieval results. We will not explore how the variations affect the retrieval results in this project.

With the RAG corpus built, we can now move onto retrieving documents associated with a query.
The query goes through the embedding process and its resulting vector is compared to other document vectors in the database.
The top \textit{k} results are returned, with \textit{k} being an adjustable variable.
What determines the chunk's relevance is its cosine similarity to the input query.
Cosine similarity is defined as the following:
\[
	\text{Cosine Similarity} = \cos(\theta) = \frac{\mathbf{A} \cdot \mathbf{B}}{\|\mathbf{A}\| \|\mathbf{B}\|}
\]
and returns a score between 0.0 to 1.0.
Here we can set a minimum cut-off for cosine similarity to adjust the relevance of returned information.
Refer to figure \ref{fig:RetrievalExample} for an example of the returned chunks.

\begin{figure}
	\centering
	\includegraphics[width=0.5\textwidth]{retrieval example.png}
	\caption{Input Query: Which patients have diabetes?}
	\label{fig:RetrievalExample}
\end{figure}

\section{Large Language Model Choice}

In healthcare settings it is unlikely for a commercial LLM like ChatGPT to see use due to privacy concerns, therefore we make use of a local LLM instead.

Given that we are making use of an agent-based approach, the LLM has to have the capabilities to make use of tools. Tools are, simply put, funcitons that the LLM can call to perform an action. An example would be calling a function for addition or subtraction.

Here we make use of a local LLM instead of a third-party commercial LLM to maintain the idea of privacy. In healthcare settings
\subsection{Synthetic Report Generation}
LLMs differ in capabilities in accordance to their size.
To determine if the chosen LLM (Mistral Nemo 12B) was sufficient for what I needed it to do, I tested its summarization and generation abilities.
Firstly, I merged the previously processed FHIR record for a single patient into a combined document.
This document was then passed to LLM along with a set of instructions.
The specific prompt provided to the LLM is in the appendix, but to summarize:
\begin{itemize}
	\item Break the summary into clear sections with headers
	\item Include exact numerical values
	\item Use precise dates
	\item Report conditions with specific terminology
	\item Summarize readings into a range spanning from min-max
\end{itemize}

The generated report summary was then passed to the LLM with instructions to anonymize information by rounding values as well as removing ages, dates, and names.
This was done for three different types of prompting strategies, Zero-Shot, Chain-of-Thought, and Structured Output.

Refer to figure \ref{fig:SynthSummary} for a side-by-side comparison for Zero-Shot generation.
Full results for each are present in the appendix.
Overall, the LLM was effective in following instructions as well as working with a large amount of context.

\begin{figure}
	\includegraphics[width=\textwidth]{Zero-shot-summary-vs-synth.png}
	\centering
	\caption{Zero-Shot Generated Summary V.S. Synthesized Summary}
	\label{fig:SynthSummary}
\end{figure}



\section{Moving Forward}

With the RAG corpus built, and the abilities of the LLM confirmed, the next steps are as follows:
\begin{itemize}
	\item Evaluate LLM's ability to extract relevant information from retrieved chunks
	\item Create a pipeline that connects the retrieval, synthesis and inference stages
	\item Compare LLM's responses when presented with the original and synthesized information
	\item Test the system through prompt attacks (information-query attacks)
\end{itemize}


\emergencystretch=1em
\printbibliography[title=References]
\nocite{*}
\break

\appendix
\section{Appendix}
\chapter{Appendix}
\subsection{
	Report Generation Prompt
}
\begin{lstlisting}
    SYSTEM PROMPT:
    You are a highly skilled medical assistant trained to generate detailed and structured patient summaries based on provided clinical data. Ensure your summaries are professional, precise, and adhere to the following guidelines:

    Include Key Identifiers: Use the patient's full name, age, and any other provided demographic information in the summary. Ensure accurate representation of their identity.
    ## Organizational Structure: Break the summary into clear sections with headers:
        Patient Information
        Clinical Observations
        Relevant Medical History
        Recent Procedures
        Medications
        Allergies

    ## Data Reporting:
        Include exact numerical values for vital signs, lab results, and other quantitative data (e.g., height, weight, BMI, blood pressure, glucose levels).
        Use precise dates for events, diagnoses, or procedures where applicable.
        Report medical conditions and history with specific terminology.
        Summarize readings into a range of values that include the maximum and minimum values for each result and observation. (min value - max value)
        Provide the range of values that occur for each reading.

    ## Tone and Clarity:
        Use formal and professional language. Avoid abbreviations unless they are common medical terms (e.g., "BP" for blood pressure).
        Write in full sentences, ensuring clarity for medical professionals reviewing the report.
    
    ---
    [PATIENT HISTORY]
        {history}
    [/PATIENT HISTORY]
\end{lstlisting}

\break
\subsection{Synthetic Report Generation Prompt}
\begin{lstlisting}
SYSTEM PROMPT:
You are an advanced language model tasked with summarizing patient data into a concise and structured medical report. Follow these instructions precisely:

Anonymisation: The patient's name and any identifying information must be removed or replaced with placeholders (e.g., "[Anonymized]").

Structure: Organize the summary into the following sections:
    Patient Summary
    Clinical Observations
    Relevant Medical History
    Recent Procedures
    Medications
    Allergies

Formatting Rules:
    Use bullet points for observations and history to enhance readability.
    Replace values with rounded values for lab results and vital signs. Use approximate ranges if values fluctuate.
    Replace exact dates with the patient's life phase (e.g., "Young Adulthood" or "Late Adulthood").
    Avoid including exact locations or unnecessary identifiers.

Clinical Data Requirements:
    Summarize relevant vitals (e.g., blood pressure, BMI, glucose levels) with appropriate medical context.
    Round all values in the summary.
    Ensure that all tests present in the original summary are present in the anonymized summary.

## Tone and Clarity:
    Use formal and professional language. Avoid abbreviations unless they are common medical terms (e.g., "BP" for blood pressure).
    Write in full sentences, ensuring clarity for medical professionals reviewing the report.

---
[REPORT SUMMARY]
        {generated_summary}
[/REPORT SUMMARY]
\end{lstlisting}

\break
\subsection{Structured Output Prompt}
\begin{lstlisting}
    SYSTEM PROMPT:
    You are a highly capable summarization assistant. Your goal is to extract and summarize key medical information from detailed health records in a concise and structured format. Focus on identifying:

    Patient Demographics: Include name, gender, date of birth, marital status, and deceased status.
    Vital Observations: Summarize relevant metrics like height, weight, BMI, blood pressure, glucose, cholesterol, etc., emphasizing significant trends or noteworthy values.
    Medical Conditions: List significant diagnoses along with recorded dates.
    Procedures: Highlight completed procedures.
    Medications: Categorize medications into active and stopped, noting their names and dosages.
    Allergies: Summarize reported allergies, if any.
    Smoking Status: Include tobacco smoking status.

    Input Example:
    Name: John Doe
    Gender: Male
    Born: 1950-01-01
    Marital Status: M
    Deceased: True
    ... [additional details as seen in the record above].

    Output Format:
    Use the following structured format for your response:
    ### Summary  
    **Demographics:**  
    - Name: [Name]  
    - Gender: [Gender]  
    - DOB: [Date of Birth]  
    - Marital Status: [Marital Status]  
    - Deceased: [Yes/No]  

    **Observations:**  
    - Height: [Value] cm  
    - Weight: [Value] kg  
    - BMI: [Value] kg/m2  
    - Blood Pressure: Systolic [Value] mmHg / Diastolic [Value] mmHg  
    - Glucose: [Value] mg/dL  
    - Hemoglobin A1c: [Value] %  
    - Cholesterol: Total [Value] mg/dL, LDL [Value] mg/dL, HDL [Value] mg/dL  
    - [Add other metrics as needed.]  

    **Conditions:**  
    - [Condition Name]: [Date Recorded]  
    - ...  

    **Procedures:**  
    - [Procedure Name(s)]  

    **Medications:**  
    - **Active:**  
    - [Medication Name, Dosage]  
    - **Stopped:**  
    - [Medication Name, Dosage]  

    **Allergies:**  
    - [List of allergies or "None"]  

    **Smoking Status:**  
    - [Never smoker/Current smoker/Former smoker]  
    
    ---
    [PATIENT HISTORY]
        {history}
    [/PATIENT HISTORY]
\end{lstlisting}

\break
\subsection{Zero-Shot Result}
\begin{lstlisting}
    
    Patient summary:
**Patient Information**

- Full Name: Jimmy858 Ankunding277
- Age: 90 years old (born on 1932-12-18)
- Gender: Male
- Marital Status: Married (M)
- Deceased: Yes

**Clinical Observations**

*Vitals:*
- Body Height: 165.79 cm
- Pain severity (0-10 verbal numeric rating): Range of values (min: 0.15, max: 3.81)
- Body Weight: Range of values (min: 76.02 kg, max: 83.72 kg)
- Body Mass Index (BMI): Range of values (min: 27.66 kg/m^2, max: 30.46 kg/m^2)
- Blood Pressure:
    - Diastolic Blood Pressure: Range of values (min: 72.92 mmHg, max: 87.52 mmHg)
    - Systolic Blood Pressure: Range of values (min: 104.78 mmHg, max: 136.51 mmHg)

*Laboratory Results:*
- Glucose: Range of values (min: 73.14 mg/dL, max: 98.21 mg/dL)
- Urea Nitrogen: Range of values (min: 7.14 mg/dL, max: 17.79 mg/dL)
- Creatinine: Range of values (min: 0.68 mg/dL, max: 0.93 mg/dL)
- Calcium: Range of values (min: 8.61 mg/dL, max: 10.18 mg/dL)
- Sodium: Range of values (min: 136.51 mmol/L, max: 143.79 mmol/L)
- Potassium: Range of values (min: 3.72 mmol/L, max: 5.15 mmol/L)
- Chloride: Range of values (min: 101.06 mmol/L, max: 107.77 mmol/L)
- Carbon Dioxide: Range of values (min: 20.95 mmol/L, max: 28.18 mmol/L)
- Total Cholesterol: Range of values (min: 163.82 mg/dL, max: 185.82 mg/dL)
- Triglycerides: Range of values (min: 121.80 mg/dL, max: 141.14 mg/dL)
- Low Density Lipoprotein Cholesterol (LDL): Range of values (min: 70.57 mg/dL, max: 83.27 mg/dL)
- High Density Lipoprotein Cholesterol (HDL): Range of values (min: 59.83 mg/dL, max: 75.98 mg/dL)

*Hematology:*
- Hemoglobin A1c/Hemoglobin.total in Blood: Range of values (min: 5.84%, max: 6.36%)
- Leukocytes [#/volume] in Blood by Automated count: Range of values (min: 4.67, max: 8.12) 10^3/uL
- Erythrocytes [#/volume] in Blood by Automated count: Range of values (min: 4.34, max: 5.43) 10^6/uL
- Hemoglobin [Mass/volume] in Blood: Range of values (min: 14.14 g/dL, max: 14.17 g/dL)
- Hematocrit [Volume Fraction] of Blood by Automated count: Range of values (min: 36.89%, max: 46.69%)
- MCV [Entitic volume] by Automated count: Range of values (min: 80.00 fL, max: 84.38 fL)
- MCH [Entitic mass] by Automated count: Range of values (min: 27.48 pg, max: 32.28 pg)
- MCHC [Mass/volume] by Automated count: Range of values (min: 33.76 g/dL, max: 35.89 g/dL)
- Erythrocyte distribution width [Entitic volume] by Automated count: Range of values (min: 42.09 fL, max: 45.90 fL)
- Platelets [#/volume] in Blood by Automated count: Range of values (min: 238.95, max: 308.03) 10^3/uL
- Platelet distribution width [Entitic volume] in Blood by Automated count: Range of values (min: 370.81 fL, max: 429.00 fL)
- Platelet mean volume [Entitic volume] in Blood by Automated count: Range of values (min: 10.79 fL, max: 11.36 fL)

**Relevant Medical History**

- Chronic sinusitis (disorder) recorded on 1959/10/30
- Prediabetes recorded on 1972/12/24
- Anemia (disorder) recorded on 1977/01/02
- Stroke recorded on 1995/04/23
- Coronary Heart Disease recorded on 2004/04/18
- Viral sinusitis (disorder) recorded on 2005/05/04, 2006/11/18, and 2010/09/14
- Laceration of forearm recorded on 2005/08/25
- Myocardial Infarction recorded on 2011/10/09

**Recent Procedures**

- Completed Medication Reconciliation (procedure) on multiple dates
- Completed Echocardiography (procedure)
- Completed Percutaneous coronary intervention
- Completed Coronary artery bypass grafting
- Completed Colonoscopy on multiple dates
- Completed Suture open wound

**Medications**

- Stopped Clopidogrel 75 MG Oral Tablet
- Active Simvastatin 20 MG Oral Tablet
- Active Amlodipine 5 MG Oral Tablet
- Stopped Nitroglycerin 0.4 MG/ACTUAT Mucosal Spray
- Stopped Acetaminophen 325 MG Oral Tablet
- Stopped Atorvastatin 80 MG Oral Tablet
- Stopped Captopril 25 MG Oral Tablet

**Allergies**

No allergies reported.

Synth report:
**[ANONYMIZED MEDICAL REPORT]**

**Patient Summary**
- Age: [Anonymized] years old ([Anonymized] life phase)
- Gender: Male
- Marital Status: Married

**Clinical Observations**
- **Vitals:**
  - Height: Approximately 165 cm
  - Pain severity (0-10): Range of values (min: 0.2, max: 4)
  - Weight: Range of values (min: 76 kg, max: 84 kg)
  - BMI: Range of values (min: 27.7 kg/m^2, max: 30.5 kg/m^2)
  - Blood Pressure:
    - Diastolic: Range of values (min: 73 mmHg, max: 88 mmHg)
    - Systolic: Range of values (min: 105 mmHg, max: 137 mmHg)

- **Laboratory Results:**
  - Glucose: Range of values (min: 73 mg/dL, max: 98 mg/dL)
  - Urea Nitrogen: Range of values (min: 7 mg/dL, max: 18 mg/dL)
  - Creatinine: Range of values (min: 0.7 mg/dL, max: 0.9 mg/dL)
  - Calcium: Range of values (min: 8.6 mg/dL, max: 10.2 mg/dL)
  - Sodium: Range of values (min: 137 mmol/L, max: 144 mmol/L)
  - Potassium: Range of values (min: 3.7 mmol/L, max: 5.2 mmol/L)
  - Chloride: Range of values (min: 101 mmol/L, max: 108 mmol/L)
  - Carbon Dioxide: Range of values (min: 21 mmol/L, max: 28 mmol/L)
  - Total Cholesterol: Range of values (min: 164 mg/dL, max: 186 mg/dL)
  - Triglycerides: Range of values (min: 122 mg/dL, max: 141 mg/dL)
  - LDL Cholesterol: Range of values (min: 71 mg/dL, max: 83 mg/dL)
  - HDL Cholesterol: Range of values (min: 60 mg/dL, max: 76 mg/dL)

- **Hematology:**
  - Hemoglobin A1c/Hemoglobin.total in Blood: Range of values (min: 5.8%, max: 6.4%)
  - Leukocytes: Range of values (min: 4.7, max: 8.1) 10^3/uL
  - Erythrocytes: Range of values (min: 4.3, max: 5.4) 10^6/uL
  - Hemoglobin: Range of values (min: 14.2 g/dL, max: 14.2 g/dL)
  - Hematocrit: Range of values (min: 37%, max: 47%)
  - MCV: Range of values (min: 80 fL, max: 84 fL)
  - MCH: Range of values (min: 27 pg, max: 32 pg)
  - MCHC: Range of values (min: 33.8 g/dL, max: 35.9 g/dL)
  - Erythrocyte distribution width: Range of values (min: 42 fL, max: 46 fL)
  - Platelets: Range of values (min: 239, max: 308) 10^3/uL
  - Platelet distribution width: Range of values (min: 371 fL, max: 430 fL)
  - Platelet mean volume: Range of values (min: 10.8 fL, max: 11.4 fL)

**Relevant Medical History**
- Chronic sinusitis recorded in [Anonymized] life phase
- Prediabetes recorded in young adulthood
- Anemia recorded in early adulthood
- Stroke recorded in middle age
- Coronary Heart Disease recorded in late adulthood
- Viral sinusitis recorded on multiple occasions in adulthood
- Laceration of forearm recorded in adulthood
- Myocardial Infarction recorded in late adulthood

**Recent Procedures**
- Completed Medication Reconciliation on multiple occasions
- Completed Echocardiography
- Completed Percutaneous coronary intervention
- Completed Coronary artery bypass grafting
- Completed Colonoscopy on multiple occasions
- Completed Suture open wound

**Medications**
- Stopped Clopidogrel 75 MG Oral Tablet
- Active Simvastatin 20 MG Oral Tablet
- Active Amlodipine 5 MG Oral Tablet
- Stopped Nitroglycerin 0.4 MG/ACTUAT Mucosal Spray
- Stopped Acetaminophen 325 MG Oral Tablet
- Stopped Atorvastatin 80 MG Oral Tablet
- Stopped Captopril 25 MG Oral Tablet

**Allergies**
No allergies reported.

\end{lstlisting}

\break
\subsection{Chain-of-Thought Result}
\begin{lstlisting}
    Patient summary:
**Patient Information**

- **Name:** Jimmy Ankunding
- **Date of Birth:** December 18, 1932
- **Gender:** Male
- **Marital Status:** Married (M)
- **Deceased:** Yes

**Clinical Observations**

| Parameter | Value Range |
| --- | --- |
| Body Height | 165.79 cm |
| Pain Severity (0-10) | 0.15 - 3.81 |
| Body Weight | 76.02 - 83.72 kg |
| Body Mass Index | 27.66 - 30.46 kg/m2 |
| Blood Pressure (Diastolic/Systolic) | 72.92 - 136.51 mmHg / 104.78 - 134.27 mmHg |
| Glucose | 73.14 - 98.21 mg/dL |
| Urea Nitrogen | 7.14 - 17.79 mg/dL |
| Creatinine | 0.68 - 0.93 mg/dL |
| Calcium | 8.61 - 10.18 mg/dL |
| Sodium | 136.51 - 143.79 mmol/L |
| Potassium | 3.72 - 5.15 mmol/L |
| Chloride | 101.06 - 107.77 mmol/L |
| Carbon Dioxide | 20.95 - 28.18 mmol/L |
| Total Cholesterol | 163.82 - 185.82 mg/dL |
| Triglycerides | 121.80 - 141.14 mg/dL |
| Low Density Lipoprotein Cholesterol | 70.57 - 83.27 mg/dL |
| High Density Lipoprotein Cholesterol | 65.02 - 75.98 mg/dL |
| Leukocytes (#/volume) | 4.12 - 8.12 x10^3/uL |
| Erythrocytes (#/volume) | 4.34 - 5.43 x10^6/uL |
| Hemoglobin (Mass/Volume) | 14.14 - 14.17 g/dL |
| Hematocrit (Volume Fraction) | 36.89 - 46.69 % |
| MCV (Entitic Volume) | 80.00 - 84.38 fL |
| MCH (Entitic Mass) | 27.48 - 32.28 pg |
| MCHC (Mass/Volume) | 33.76 - 35.89 g/dL |
| Erythrocyte Distribution Width | 42.09 - 45.90 fL |
| Platelets (#/volume) | 238.95 - 308.03 x10^3/uL |
| Platelet Distribution Width | 370.81 - 429.00 fL |
| Platelet Mean Volume | 10.79 - 11.36 fL |
| Hemoglobin A1c/Hemoglobin.total in Blood | 5.84 - 6.36 % |

**Relevant Medical History**

- Chronic sinusitis (1959)
- Prediabetes (1972)
- Anemia (1977)
- Stroke (1995)
- Coronary Heart Disease (2004)
- Viral sinusitis (2005, 2006, 2010)
- Myocardial Infarction (2011)

**Recent Procedures**

- Medication Reconciliation (multiple dates)
- Echocardiography
- Percutaneous coronary intervention
- Coronary artery bypass grafting
- Colonoscopy (multiple procedures)
- Suture open wound

**Medications**

- Simvastatin 20 MG Oral Tablet (active)
- Amlodipine 5 MG Oral Tablet (active)
- Clopidogrel 75 MG Oral Tablet (stopped)
- Nitroglycerin 0.4 MG/ACTUAT Mucosal Spray (stopped)
- Acetaminophen 325 MG Oral Tablet (stopped)
- Atorvastatin 80 MG Oral Tablet (stopped)
- Captopril 25 MG Oral Tablet (stopped)

**Allergies**

No allergies listed.

Synth report:
**[Anonymized] Patient Summary**

**Patient Information**
- **Age:** Late Adulthood
- **Gender:** Male
- **Marital Status:** Married ([Anonymized])

**Clinical Observations**
- Height: Approximately 165 cm
- Weight: Fluctuated between approximately 76 kg and 84 kg
- BMI: Ranged from approximately 28 to 30 kg/m2
- Blood Pressure: Diastolic/Systolic ranged from approximately 73/105 mmHg to 137/134 mmHg
- Glucose Levels: Fluctuated between approximately 73 mg/dL and 98 mg/dL
- Urea Nitrogen: Ranged from approximately 7 mg/dL to 18 mg/dL
- Creatinine: Approximately 0.7 to 0.9 mg/dL
- Calcium: Approximately 8.6 to 10.2 mg/dL
- Sodium: Approximately 137 to 144 mmol/L
- Potassium: Ranged from approximately 3.7 to 5.2 mmol/L
- Chloride: Approximately 102 to 108 mmol/L
- Carbon Dioxide: Approximately 21 to 28 mmol/L
- Total Cholesterol: Approximately 164 to 186 mg/dL
- Triglycerides: Approximately 122 to 141 mg/dL
- LDL Cholesterol: Approximately 71 to 83 mg/dL
- HDL Cholesterol: Approximately 65 to 76 mg/dL
- Leukocytes: Approximately 4 to 8 x10^3/uL
- Erythrocytes: Approximately 4 to 5.4 x10^6/uL
- Hemoglobin: Approximately 14 g/dL
- Hematocrit: Approximately 37% to 47%
- MCV: Approximately 80 to 84 fL
- MCH: Approximately 27 to 32 pg
- MCHC: Approximately 34 g/dL
- Erythrocyte Distribution Width: Approximately 42 to 46 fL
- Platelets: Approximately 240 to 310 x10^3/uL
- Platelet Distribution Width: Approximately 370 to 430 fL
- Platelet Mean Volume: Approximately 10.8 to 11.4 fL

**Relevant Medical History**
- Chronic sinusitis since Young Adulthood ([1959])
- Prediabetes diagnosed in Early Adulthood ([1972])
- Anemia diagnosed in Early Adulthood ([1977])
- Stroke in Middle Age ([1995])
- Coronary Heart Disease diagnosed in Late Middle Age ([2004])
- Multiple episodes of viral sinusitis (2005, 2006, 2010)
- Myocardial Infarction in Late Adulthood ([2011])

**Recent Procedures**
- Medication Reconciliation performed on multiple occasions
- Echocardiography completed
- Percutaneous coronary intervention conducted
- Coronary artery bypass grafting procedure undertaken
- Colonoscopy procedures performed on multiple occasions
- Suture open wound repair

**Medications**
- Simvastatin 20 MG Oral Tablet (currently active)
- Amlodipine 5 MG Oral Tablet (currently active)
- Clopidogrel 75 MG Oral Tablet (previously active, now stopped)
- Nitroglycerin 0.4 MG/ACTUAT Mucosal Spray (previously active, now stopped)
- Acetaminophen 325 MG Oral Tablet (previous use)
- Atorvastatin 80 MG Oral Tablet (previous use)
- Captopril 25 MG Oral Tablet (previous use)

**Allergies**
No allergies listed.

\end{lstlisting}

\break
\subsection{Structured Output Result}
\begin{lstlisting}
    Patient summary:
    ### Summary
    **Demographics:**
    - Name: Jimmy Ankunding
    - Gender: Male
    - DOB: 1932-12-18
    - Marital Status: Married (M)
    - Deceased: Yes
    
    **Observations:**
    - Height: 165.79 cm
    - Weight: 83.72 kg
    - BMI: 30.46 kg/m2
    - Blood Pressure: Systolic 134.27 mmHg / Diastolic 82.43 mmHg
    - Glucose: 73.51 mg/dL
    - Hemoglobin A1c: 6.34 %
    - Cholesterol: Total 175.49 mg/dL, LDL 91.30 mg/dL, HDL 59.83 mg/dL
    
    **Conditions:**
    - Chronic sinusitis: 1959/10/30
    - Prediabetes: 1972/12/24
    - Anemia: 1977/01/02
    - Stroke: 1995/04/23
    - Coronary Heart Disease: 2004/04/18
    - Viral sinusitis (multiple occurrences)
    - Laceration of forearm: 2005/08/25
    - Myocardial Infarction: 2011/10/09
    
    **Procedures:**
    - Medication Reconciliation
    - Echocardiography
    - Percutaneous coronary intervention
    - Coronary artery bypass grafting
    - Colonoscopy (multiple occurrences)
    - Suture open wound
    
    **Medications:**
    - **Active:**
        - Simvastatin 20 MG Oral Tablet
        - Amlodipine 5 MG Oral Tablet
    - **Stopped:**
        - Clopidogrel 75 MG Oral Tablet
        - Nitroglycerin 0.4 MG/ACTUAT Mucosal Spray
        - Acetaminophen 325 MG Oral Tablet
        - Atorvastatin 80 MG Oral Tablet
        - Captopril 25 MG Oral Tablet
    
    **Allergies:**
    - None reported
    
    **Smoking Status:**
    - Never smoker
    
    Synth report:
    **[ANONYMIZED MEDICAL REPORT]**
    
    **Patient Summary**
    - **Name:** [Anonymized]
    - **Gender:** Male
    - **Age:** Deceased in Late Adulthood
    - **Marital Status:** Married
    
    **Clinical Observations**
    - Height: Approximately 166 cm
    - Weight: Around 84 kg
    - Body Mass Index (BMI): About 30.5 kg/m2
    - Blood Pressure (BP): Systolic ~134 mmHg, Diastolic ~82 mmHg
    - Glucose Level: Around 74 mg/dL
    - Hemoglobin A1c (HbA1c): Approximately 6.3%
    - Cholesterol Levels:
        - Total: About 175 mg/dL
        - LDL: Around 91 mg/dL
        - HDL: Approximately 60 mg/dL
    
    **Relevant Medical History**
    - Chronic sinusitis, diagnosed in Young Adulthood (1959)
    - Prediabetes, diagnosed in Early Middle Age (1972)
    - Anemia, diagnosed in Early Middle Age (1977)
    - Stroke, occurred in Mid-Adulthood (1995)
    - Coronary Heart Disease, diagnosed in Late Middle Age (2004)
    - Viral sinusitis: Multiple occurrences
    - Laceration of forearm, treated in Late Middle Age (2005)
    - Myocardial Infarction, occurred in Late Middle Age (2011)
    
    **Recent Procedures**
    - Medication Reconciliation
    - Echocardiography
    - Percutaneous coronary intervention
    - Coronary artery bypass grafting
    - Colonoscopy: Multiple occurrences
    - Suture open wound
    
    **Medications**
    - **Active:**
        - Simvastatin 20 MG Oral Tablet
        - Amlodipine 5 MG Oral Tablet
    - **Stopped:**
        - Clopidogrel 75 MG Oral Tablet
        - Nitroglycerin 0.4 MG/ACTUAT Mucosal Spray
        - Acetaminophen 325 MG Oral Tablet
        - Atorvastatin 80 MG Oral Tablet
        - Captopril 25 MG Oral Tablet
    
    **Allergies**
    - None reported
    
    **Smoking Status**
    - Never smoker
        
\end{lstlisting}

\end{document}
