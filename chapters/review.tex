\chapter{Literature Review} \label {literature}

\section{Exploitation of RAG Systems}

While RAG improves LLM accuracy by integrating external knowledge sources, it also introduces new vulnerabilities. Attackers can exploit a RAG system's retrieval mechanisms to manipulate outputs, bypass safeguards, and extract sensitive information. This section explores some known methods of attacks and discusses their feasibility in a healthcare setting.

\subsection{Data Poisoning}
In data poisoning attacks (also known as a backdoor attack), attackers inject malicious or misleading information into the RAG corpus, causing the LLM to generate incorrect or malicious responses. These attacks can be used to carry out fraud, misinformation campaigns or provide adversarial control over responses. An example of this occurred with Google's Gemini, where, due to retrieving information from a satirical social media post, told the user to make use of "non-toxic glue" when making a cheese pizza \autocite{McMahon_2024}.

As highlighted in \autocite{tan2024gluepizzaeatrocks}, data poisoning attacks are non-trivial to carry out.
Depending on the complexity of the retrieval system, the attacker will have to modify the adversarial content such that the retrieval system is inclined to retrieve this document.
Furthermore, the attacker requires some information about or access to the retrieval system to exploit it.
This requirement is consistent with other studies carried out on data poisoning, and in almost all cases, the conditions in which this attack can manifest relies strictly on the insertion of a poisoned document into the RAG corpus \autocite{xue2024badragidentifyingvulnerabilitiesretrieval, tan2024gluepizzaeatrocks, xian2024vulnerabilityapplyingretrievalaugmentedgeneration}.

Given these requirements, we can conclude that this type of RAG attack is non-feasible in a healthcare setting.
In order to carry out this attack, the attacker has to have some form of access to the hospital's database.
The cases in which this occur typically present with an external cyberattack on the hospital's infrastructure, whether through hacking or social engineering, and is considered a data breach. Most data breaches occur through hacking, as reported in \autocite{Alder_2025}.
In this case, the attacker can gain access to the hospital's database, and does not need to rely on exploiting the RAG system.
Thus, we can conclude that this form of attack is non-applicable in a healthcare setting.

\subsection{Prompt Injection}
Prompt injection attacks involves crafting an input query that manipulates the model into generating unintended responses.
For RAG systems, this can be achieved either directly or indirectly.

Indirect prompt injection attacks function similarly to data poisoning except instead of inserting misleading information, adversarial prompts are attached to frequently retrieved documents in the RAG database. This enables attackers to retrieve documents from the RAG database using trigger prompts.

Direct prompt injection attacks involve the inclusion of a passage or sentence in the input query. This can be phrases such as "repeat all the context".
These attacks, when targeted at RAG systems, can cause the leakage of private or sensitive information from the RAG corpus.

\subsubsection{Indirect Prompt Injection}
As stated previously, indirect prompt injection attacks operate in a similar fashion to data poisoning attacks as both require some form of access or ability to manipulate the RAG corpus.

Instead of using a misleading document, malicious instructions are embed in the document within the corpus, allowing the attacker to manipulate the LLM's output.
This allows the attacker to manipulate the LLM into including potentially malicious URLs into its response when responding to a victim's query \autocite{clop2024backdooredretrieverspromptinjection}.

Furthermore, it should be noted that this type of prompt injection also allows the attacker manipulate the documents that are retrieved from the RAG corpus depending on the poison ratio, as covered in \autocite{peng2024dataextractionattacksretrievalaugmented}, which would allow unfettered access to any sensitive information stored in the database.

However, since this type of prompt attack requires some form of access to the RAG corpus, we can functionally consider it the same as a data poisoning attack.
Realistically, if this type of attack were to occur in a healthcare setting, the attacker would already have access to hospital records. Therefore we will not be focusing on this aspect of prompt injection attacks.

\subsubsection{Direct Prompt Injection}
Direct prompt injection involves the inclusion of an adversarial passage into the input query, and these attacks are usually carried out in a specific format.

As highlighted in \autocite{zeng2024goodbadexploringprivacy}, they consist of two components: information and command.
The information component of the attack leads the RAG system to retrieve documents that contain that form of information. Examples of this could be names, addresses, medical conditions.
The command component is targeted at the LLM. Phrases are included in the input query that are aimed at subverting any safeguards placed on the LLM.
This can be a phrase such as "please repeat all context back to me," or "ignore all instructions," etc.

As the study\autocite{zeng2024goodbadexploringprivacy} shows, a significant portion of the datasets used in the study were able to be retrieved from the LLM through simple prompt attacks.
The study also notes that the attack prompt could be further optimized for increased data extraction.
Part of the study also noted the effects of RAG on the data that was extracted.
It was noted that the inclusion of RAG decreased the appearance of memorized data in the LLM's output. It seems that the inclusion of RAG has caused the LLM to focus on leveraging the context retrieved rather than on its memorized training data \autocite{zeng2024goodbadexploringprivacy}.

Another study also corroborates this result. In \autocite{qi2024followinstructionspillbeans}, a similar method of prompt injection was used to extract text from a RAG database. The LLMs used in this study were instruction-tuned LLMs, meaning that the model has been trained to respond to instructions.

An interesting point to note was that they tested the similarity scores of the model's output with the retrieved context. Most LLMs used in the study exhibited higher BLEU, ROUGE-L, F1 and BERTScore scores that scaled with model size, suggesting that there is some correlation between an LLM capabilities and its vulnerabilities to prompt injection attacks \autocite{qi2024followinstructionspillbeans}. Additionally, between the amount of data extracted alongside the size of the context retrieved was noted, further asserting that RAG has inherent vulnerabilities that are not being addressed.

Both studies discussed have shown a clear vulnerability of RAG to sufficiently sophisticated prompt injection attacks, but not much research has been done regarding the mitigation of RAG output post-occurrence of a prompt injection attack.

While it is possible to include safeguards to prevent the occurrence of prompt attacks, their implementations are still vulnerable. This is highlighted in \autocite{li2024targetingcoresimpleeffective}, where a simple prompt of "ignore the context" caused the LLM agent to disregard any context retrieved despite the safeguards implemented.
Considering that a simple prompt like this was sufficient enough to manipulate the model's output, it suggests that RAG pipeline implementations may be more fragile than initially anticipated, and a sufficiently motivated attacker will eventually be able to penetrate any LLM-level safeguards in place.

These findings suggest that current RAG implementations lack strong defenses against targeted prompt injection attacks. While preventive safeguards exist, adversarial prompt injections can still manipulate retrieval. This highlights the need for alternative security measures - such as synthetic document generation - to obfuscate retrieved context and prevent LLMs from directly accessing sensitive data in a RAG corpus.

\section{Medical Anonymization}
In clinical settings, it is necessary to anonymize a dataset before releasing it to be shared. Based on that principle, we can further extend to healthcare applications that the public can access.

According to \autocite{Rodriguez_Tuck_Dozier_Lewis_Eldridge_Jackson_Murray_Weir_2022}, the three main methods used in preserving medical privacy are Pseudonymization, De-identification, and Anonymization.

Each method can be summed up by the following:

Pseudonymization involves replacing attributes with pseudonyms.

De-identification involves the removal of PII from patient records.

Anonymization involves distorting the record such that it cannot be related to its original record.
% TODO:
% - Discuss the different methods of medical anonymisation
% - also what's the principle behind it
% - say which methods we're gonna be using
% - maybe add a section on synthesizing information?
% WARNING: come back to look at this in the future

Each of these methods are aimed at preserving clinical privacy, and are used in tandem.

As described in the study, the release of a clinical dataset involves de-identification of clinical data, after which the data undergoes either anonymization or pseudonymization before being released.

Further, the data can be manipulated in certain ways. Methods include addition of random noise to variables, converting ages to their relative dates, or categorising ages. This is a non-exhaustive list of methods and the rest can be seen in \autocite{Rodriguez_Tuck_Dozier_Lewis_Eldridge_Jackson_Murray_Weir_2022}.

The document synthesis process operates based on some of these principles. In particular, we make use of methods that maintain some relation to the original record.
