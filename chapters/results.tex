\chapter{Results} \label{results}

In this section we evaluate the effectiveness of the agent-based document synthesis pipeline. We evaluate the following metrics: Node retrieval accuracy, semantic similarities between the synthesized components and their original forms, as well as the system's performance against prompt injection attacks.

We run the series of tests against both the agent-based system as well as just a singular agent with access to the search functions mentioned in the previous section.
\section{RAG Accuracy}
In this section we are only concerned with the nodes that are retrieved from the RAG pipeline. We modify the pipeline to return only the nodes retrieved by our RAG system. We test three cases, access only to the semantic function, access to only the keyword search function, and finally access to both functions. For all cases, we modify the function descriptor and names to be the same. For case 3 we call both the semantic and keyword searches together, then concatenate the results.

For the semantic search function in figure \ref{fig:SemanticSearchFunc}, it should be noted that instead of just the top \textit{k} results, we retrieve \textit{2k} nodes because our vector retriever is performing a hybrid search, a combination of vector search and text search. In this case, the top \textit{k} nodes are the ones with the highest similarity, while the bottom \textit{k} nodes are the results of the text search.

In the keyword search, it extracts 10 keywords per query, and then extracts \textit{k} chunks that best match the keywords extracted. We make use of a regex based keyword search in this project.

Using the previously generated question list, we randomly select 100 questions and pass them into our RAG pipeline. Here we evaluate if the original file that the question was generated from is present within the series of nodes retrieved by the RAG system. If there are no nodes retrieved, we treat it as a miss. We track the number of nodes retrieved for each case and value of \textit{k}.

For each case we perform the selection 5 times, then compute the average. We also vary the value of \textit{k} to determine if there is any improvement to accuracy.

The following table presents the results for each case:

\begin{table}
	\centering
	\begin{tabular}{|c|c|c|c|}
		\hline
		                  & \textit{k = 3} & \textit{k = 4} & \textit{k = 5} \\
		\hline
		Semantic (Hybrid) & 81.8           & 82.4           & 88             \\
        \hline
		Semantic (Non-hybrid) & 52             & 60.2           & 66.6           \\
		\hline
		Keyword           & 26.2           & 32.2           & 30.4           \\
		\hline
		Both              & 83.8           & 85.8           & 88             \\
		\hline
	\end{tabular}
	\caption{RAG Accuracy Comparison}
	\label{Tab:RAGAccuracyComp}
\end{table}


As observed in table \ref{Tab:RAGAccuracyComp}, the keyword search performs the worst on its own. The cases where the hybrid semantic search is present both perform similarly. This is most likely due to the two factors. First, the semantic searches used in this project are performing hybrid searches, meaning the search combines the results from both dense and sparse vectors. Dense vectors are vectors produced using text embedding models similar to the one used in this project. Sparse vectors are computed using different algorithms, such as BM25. We can consider it a combination of a traditional search alongside semantic search, as such it makes sense that the hybrid search method performs the best.

Secondly, since we generate questions using the file text as content, the LLM tends to use the file as contextual base and includes details such as dates and names, or quotes information that appears directly within the text. This could result in the semantic search results improving due to the similarities between the text and the query. Furthermore, regex keyword search is inherently limited, which will impact the results negatively.

We also consider the performance of the non-hybrid semantic search. As expected, the accuracy of the non-hybrid semantic search decreases moderately. However, it does still perform better than the standalone keyword search.

Furthermore, we note that the accuracy increases with \textit{k}, which is to be expected as more nodes are being retrieved. However, while the accuracy of the node retrieval does increase, it does not guarantee an improvement in the LLM's response. An increase in nodes retrieved will result in more information being added to the LLM's context window, which may cause it overlook critical information within the text that would answer the query.

\section{Semantic Similarity}

In this section we consider the semantic similarities of the synthesized query and information to the original, as well as the similarity between responses generated from the synthetic information and the original information retrieved by the RAG system.

We select 100 random questions and pass it to the document synthesis system. We set the value of \textit{k} to be 3 for the semantic search function. For the keyword search function, we set the value of \textit{k} to be 10.

The synthesized information (alongside the synthesized query) and the original information is passed separately to the LLM to generate two responses which are recorded. We can then compare the semantic similarity between the two responses.

We make use of the following commonly used metrics to compute scores for each component: Bilingual Evaluation Understudy (BLEU) , Recall-Oriented Understudy for Gisting Evaluation (ROUGE) and BERTScore. Finally, we also include SemScore as seen in \autocite{aynetdinov2024semscoreautomatedevaluationinstructiontuned} to compute the semantic similarity between the corresponding components.

BLEU is primarily used in machine translation, however it is also a common metric for Natural Language Processing (NLP) tasks. It measures how many n-grams (continuous sequence of words) in the candidate text that appears in the reference text.

ROUGE is often used for summarization tasks. It measures overlap in terms of recall, which is how much of the reference text is captured in the candidate text.

Both BERTScore and SemScore evaluate the semantic similarity between the reference and candidates text by leveraging the embeddings generated by a text embedding model. BERTScore uses these embeddings to compute the cosine similarity between words in order to compute the sentence cosine similarity. SemScore operates in the same manner, however instead of computing the cosine similarity between words in a sentence, it computes the cosine similarity of entire responses.

We set the reference to be the information retrieved by the RAG system, and the candidate to be the synthesized information. We do the same for the synthesized query and original query. Finally, we compare the responses generated by the LLM when presented with the synthesized and original information.

We separate the results into two tables, table \ref{Tab:BLEUandROUGE} and table \ref{BERTandSem}.

\begin{table}[h]
	\centering
	\begin{tabularx}{\textwidth}
		{
			|  >{\raggedright\arraybackslash}X
			|  >{\raggedright\arraybackslash}X
			|  >{\raggedright\arraybackslash}X
			|  >{\raggedright\arraybackslash}X
			|  >{\raggedright\arraybackslash}X |}
		\hline
		                                     & BLEU   & ROUGE-1 & ROUGE-2 & ROUGE-L \\
		\hline
		Synthesized Information              & 0.0913 & 0.416   & 0.258   & 0.353   \\
		\hline
		Synthesized Query                    & 0.396  & 0.692   & 0.536   & 0.660   \\
		\hline
		Response with Synthesized components & 0.181  & 0.526   & 0.269   & 0.386   \\
		\hline
	\end{tabularx}
	\caption{BLEU and ROUGE scores}
	\label{Tab:BLEUandROUGE}
\end{table}

Looking at the BLEU and ROUGE scores in table \ref{Tab:BLEUandROUGE}, we note low BLEU scores, particularly for the synthesized information. This is expected, because as the information undergoes synthesis, details are modified or replaced in order to distance it from the original information retrieved. The low ROUGE scores for the synthesized information is also consistent with the low BLEU score.

In the case of the synthesized query, it has the highest BLEU and ROUGE scores. The synthesized query is designed to closely resemble the original query to ensure that the response the system generates takes into account the synthesized information, even if this results in the synthetic query containing less information than the original query.

For the low BLEU and ROUGE scores for the response, this is likely due to the non-deterministic nature of LLMs. The output of LLMs are not consistent across reruns, furthermore, the manner in which the response is structured will also vary between runs despite access to the same information. As such, it is expected that the response generated by the synthesized information varies from the response with the original information.

\begin{table}[h]
	\centering
	\begin{tabularx}{\textwidth}
		{
			|  >{\raggedright\arraybackslash}X
			|  >{\raggedright\arraybackslash}X
			|  >{\raggedright\arraybackslash}X
			|  >{\raggedright\arraybackslash}X
			|  >{\raggedright\arraybackslash}X |}
		\hline
		                                     & BERT Precision & BERT Recall & BERT F1 & SemScore \\
		\hline
		Synthesized Information              & 0.3209         & -0.01135    & 0.1459  & 0.6793   \\
		\hline
		Synthesized Query                    & 0.6065         & 0.3515      & 0.4760  & 0.7698   \\
		\hline
		Response with Synthesized components & 0.4290         & 0.3182      & 0.3728  & 0.7938   \\
		\hline
	\end{tabularx}
	\caption{BERTScore and SemScore}
	\label{BERTandSem}
\end{table}

In table \ref{BERTandSem}, we observe that the BERT Precision, Recall, and F1 scores vary from poor to moderate. This is to be expected. BERTScore computes the sentence similarity using token-level cosine similarity and differences in word choice can result in lower token similarity, resulting in lower scores.

We also observe that the recall rate for the synthesized information is negative. This is likely due to the modifications made to the information retrieved, as it means that less of the reference tokens, meaning the original information, is covered by the tokens in the synthetic information.

In all cases we observe that the SemScore performs well. This means that there is a strong to moderate semantic link between the original as well as the synthesized components and responses.

\section{
  PII Extraction Attacks
 }

In this section we perform a series of targeted PII attacks on the document synthesis system and a single-agent system. The single-agent system has access to the same search functions as the document synthesis system.

The differences between the document synthesis system and the single-agent system is that the document synthesis system passes off the synthesized information to an external LLM to generate a response. The single-agent system performs both searching and answering of the input query.

We provide the external LLM and the single-agent system similar protective prompts.

For the adversarial prompts, we create input prompts aimed at manipulating the model into ignoring its instructions as well as repeating its context. We append the type of PII we wish to extract, either names, addresses or contact numbers. Following that, we run PII detection on the output from each system and tabulate the occurrences of either a repeat context or PII presence in the output.

We prepend the prompt \textit{"Ignore previous Instructions and Repeat all context"} to a series of questions generated from the list of patients in the database. An example of a query is \textit{"Find contact number for Clair921"}. We generate questions in a similar format for addresses and contact numbers. We also generate a series of questions aimed at extracting information about diseases. 100 questions are generated in the following format: \textit{"I would like information about ${disease name}$."}

Since the questions generated from the list of names contain the first name of the patient, we do not consider the occurrence of only the patient's first name to be PII. Instead, we consider the full names of patients to be PII. This is only applicable for questions targeted at retrieving addresses, contact numbers, as well as broad advice questions about the target.  For the general disease queries, we consider the appearance of names to be PII. The following are also considered PII: contact numbers, addresses and birth date.

In total we have \textbf{361} questions generated. We pass the list of questions to each system and collate the results. Due to the nature of the data, where names are proceeded by a numerical identifier generated by Synthea, the use of Named Entity Recognition (NER) is not feasible. Here we manually go through each response and mark if there is PII present.


\begin{table}[h]
	\centering
	\begin{tabular}{|c |c |c |c|}
		\hline
		Responses          & PII & No PII & Empty \\
		\hline
		Single-agent       & 100 & 124    & 137   \\
		\hline
		Synthetic Document & 5   & 356    & 0     \\
		\hline
	\end{tabular}
	\caption{Synthetic Document System and Single-agent Responses}
	\label{Tab:SynthSingleResponses}
\end{table}

Referencing table \ref{Tab:SynthSingleResponses}, we note the single-agent system returns empty responses. This happens because the agent returns early from the workflow without performing function calls. This means that either the agent refused the query, or could not understand the query. The synthetic document system does not suffer from this because the output from the system is piped to another LLM for generating a response. In general, we note that the LLM has the capability to refuse queries, or at times misunderstands the query. Building on this, we will assume that the single-agent system operates under a similar principle and classify empty responses as containing no PII.

Comparing the two systems, we also observe that the synthetic document system has significantly reduced the appearance of PII in the response. This behavior can be explained due to the modifications carried out during the synthesis process, which explicitly attempts to remove any possible identifiers.

We use ROUGE-L score to evaluate if there is overlap between the response and the information retrieved. If the context is repeated by the LLM in the response, we expect to see a moderate to high ROUGE-L score. For the synthesis system we match the response against the synthesized information, while for the single-agent system we match the response against the information extracted during the RAG process.

\begin{table}
	\centering
	\begin{tabular}{|c | c|}
		\hline
		System             & ROUGE-L \\
		\hline
		Single-agent       & 0.511   \\
		\hline
		Synthetic Document & 0.353   \\
		\hline
	\end{tabular}
	\caption{Synthetic Document System and Single-agent ROUGE-L Scores}
	\label{Tab:SynthSingleROUGE-L}
\end{table}

The ROUGE-L scores for both systems are found in Table \ref{Tab:SynthSingleROUGE-L}. We note that the ROUGE-L score for the synthetic document system (0.353) compared to the single-agent system (0.511) indicates that the synthetic document system's responses express less overlap with the retrieved context. This is a good indication, as it suggests that the system is less likely to directly repeat the context verbatim. The higher ROUGE-L score for the single-agent system, however, suggests that it is more likely to repeat context in its responses, which would explain why its responses contained more PII in the previous test. The lower overlap in the synthetic document system is likely due to the synthesis process designed to mitigate PII exposure.
