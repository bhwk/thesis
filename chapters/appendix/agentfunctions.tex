\chapter{Functions for Agents} \label{AgentFunctions}

\begin{figure}[h]
	\centering
	\small
	\begin{lstlisting}[language=Python, breaklines=true]
async def record_information(ctx: Context, information: str) -> str:
    """Useful for recording information for a given query. Your input should be information written in plain text."""
    current_state = await ctx.get("state")
    if "information" not in current_state:
        current_state["information"] = []
    current_state["information"].append(information)
    await ctx.set("state", current_state)

    return "Information recorded."
    \end{lstlisting}
	\caption{Information Function}
\end{figure}

\begin{figure}[h]
	\centering
	\small
	\begin{lstlisting}[language=Python, breaklines=true]
async def review_response(ctx: Context, review: str) -> str:
    """Useful for reviewing a response and providing feedback. Your input should be a review of the report."""
    current_state = await ctx.get("state")
    if "review" not in current_state:
        current_state["review"] = ""
    current_state["review"] = review

    await ctx.set("state", current_state)
    return "Response reviewed."
    \end{lstlisting}
	\caption{Review Function}
\end{figure}


\begin{figure}[h]
	\centering
	\small
	\begin{lstlisting}[language=Python, breaklines=true]
async def synthesize_information(ctx: Context, synthesized_information: str) -> str:
    """Useful for creating synthetic context from base information provided. Your input should be the synthesized information"""
    current_state = await ctx.get("state")
    if "synthesized_information" not in current_state:
        current_state["synthesized_information"] = ""
    current_state["synthesized_information"] = synthesized_information

    await ctx.set("state", current_state)
    return "Content generated."
    \end{lstlisting}
	\caption{Synthesize Information Function}
\end{figure}

\begin{figure}[h]
	\centering
	\small
	\begin{lstlisting}[language=Python, breaklines=true]
async def synthesize_query(ctx: Context, synth_query: str) -> str:
    """Useful for creating a synth query based from the original query. Your input should be a generated, synthesized version of the user's query."""
    current_state = await ctx.get("state")
    if "synth_query" not in current_state:
        current_state["synth_query"] = ""
    current_state["synth_query"] = synth_query

    await ctx.set("state", current_state)
    return "Query generated."
    \end{lstlisting}
	\caption{Synthesize Query Function}
\end{figure}

\begin{figure}[h]
	\centering
	\small
	\begin{lstlisting}[language=Python, breaklines=true]
async def record_nodes(ctx: Context, nodes: list[NodeWithScore]) -> str:
    """Useful for recording the nodes retrieved from a search. Your input should be the list of nodes retrieved"""
    current_state = await ctx.get("state")
    if "nodes" not in current_state:
        current_state["nodes"] = []
    current_state["nodes"].extend(nodes)

    await ctx.set("state", current_state)
    return "Nodes recorded"
    \end{lstlisting}
	\caption{Record Nodes Function}
\end{figure}


\begin{figure}[h]
	\centering
	\small
	\begin{lstlisting}[language=Python, breaklines=true]
async def generate_response(ctx: Context, response: str) -> str:
    """Used to generate a response to a user's query using the information retrieved."""

    current_state = await ctx.get("state")
    current_state["response"] = response

    await ctx.set("state", current_state)

    return "Response written."
    \end{lstlisting}
	\caption{Generate Response Function}
\end{figure}
